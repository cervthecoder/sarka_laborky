\documentclass[13pt, a4paper, twoside]{article}
\usepackage[utf8]{inputenc}
\usepackage{geometry}
\usepackage[czech]{babel}
\usepackage{chemformula}
\usepackage{chemfig}
\usepackage{enumitem}
\usepackage{float}
\usepackage{fancyhdr}
\usepackage{setspace}
\usepackage{multicol}
\geometry{legalpaper, margin=1.05in}
\pagestyle{fancy}
\lhead{\Large Šárka Doležalová, skupina 6}
\rhead{\large 30.6.2020}
\begin{document}
    \begin{center}
        \Huge
        Úloha 1: Krystalizace
    \end{center}
    \large
\onehalfspacing
    \section*{Zadané úlohy}
    \begin{enumerate}
        \item Překrystalizujte vzorek acetanilidu z horké vody.
        \item Překrystalizujte vzorek acetanilidu z horkého toluenu.
        \item Překrystalizujte vzorek síranu měďnatého srážením vodného roztoku ethanolem.
    \end{enumerate}
    \section*{Teoretický úvod}
    \subsection*{Zahřívání}
    Metody, které lze využít k zahřívání reakčních směsí jsou kahany, topná hnízda,
    topné míchačky nebo lázně, které jsou zahřívané na topných plotýnkách a v některých případech
    i nad kahanem. K žíhání, neboli zahřívání vzorku, či reakční směsi na velmi vysokou
    teplotu, užíváme žíhacích pecí, ve kterých teplota vroste až nad $1000^{\circ}C$.

    Při našich postupech je důležité předcházet nebezpečí utajeného varu, neboli překročení
    teploty varu rozpouštědla a následnému vykypění reakční směsi. Tomuto nežádoucímu jevu
    lze zabránit mícháním, ke které mu využíváme magnetových míchátek, nebo varných kamínků (nelze je použít opakovaně).
    \subsection*{Krystalizace}
    Nejužívanější metodou pro čištění vzorků v chemické laboratoři je krystalizace.
    Vzorek, neboli znečištěnou pevnou látku rozpustíte v rozpouštědle a následně je snížena
    její rozpustnost, čehož docílíme ku příkladu změnou teploty a z roztoku se vyloučí látka s
    vyšší čistotou, než měla před provedením krystalizace. Krystalizaci lze rozlišit na tři
    druhy, při kterých je užíváno rozdílných postupů.
    \subsubsection*{Krystalizace srážením}
    Krystalizace srážením, nebo také krystalizace změnou rozpouštědla, je dosažena přidáním
    k látce s rozpuštědlem druhého rozpouštědla, ve kterém látka není rozpustná, ale je mísitelné
    s původním rozpouštědlem.
    \subsubsection*{Krystalizace odprařením rozpouštědla}
    Krystalizace odpařením rozpouštědla, neboli krystalizace zahuštěním pracuje na principu
    částečného odpaření rozpouštědla, což vede k zahuštění směsi. Nevýhodou této metody je,
    že čištěná látka je po celou dobu procesu ve styku s matečným roztokem, nebo může dojít k rozkladu
    produktu.
    \subsubsection*{Krystalizace změnou teploty}
    Krystalizace změnou teploty je založená na tom, že rozpustnost látek v rozpuštědle se
    v závislosti na teplotě mění. Ve většině případech se stane to, že se z roztoku při snížení
    teploty vyloučí část látky, ale v některých situacích k tomuto jevu dojde při zvýšení teploty.
    Takovéto anomální chování vykazuje například hydroxid vápenatý, nebo chroman vápenatý.
    \subsection*{Filtrace a odsávání}
    Metody filtrace a odsávání jsou využívány v chemické laboratoři k oddělení pevné fáze
    směsi od roztoku. V závislosti na velikosti zrn pevné fáze směsi zvolíme metodu. Velká
    zrna oddělujeme ku příkladu i pouze přes smotek vaty. Na druhé straně jemné sraženiny, či aktivní
    uhlí etc., je potřeba filtrovat přes filtrační papír.

    Odsávání je prováděno za sníženého tlaku, a to buď přes filtrační papír na Büchnerově či Hirschově nálevce, nebo
    na fritě.

    \section*{Postup}

    \subsection*{Rekrystalizace acetanilidu z horké vody}
    Surový acetanilid o hmotnosti $3.00 \: g$ byl rozpuštěn v Erlenmeyeřově baňce o objemu
    $250 \: ml$ v zhruba $80\: ml$ vody. Roztok byl postaven na topnou míchačku a bylo přidáno magnetické míchátko. Do hrdla baňky byla vložena nálevka, která předešla na nadměrnému
    odparu a zároveň sloužila jako vzdušný chladič. Za neustálého míchání byla směs přivedena
    k varu a cca 2 minuty byla povařena. Do připravené $100 \: ml$ baňky s předehřátou nálevkou,
    do niž byl vložen skládaný filtrový papír, byla přefiltrována směs a baňka byla zazátkována,
    umyta pod proudem studené vody a vložena na cca 20 minut do ledničky. Poté byl produkt odsát
    na fritě. Produkt byl promyt střičkou pomocí malého množtví vody. Následně byl produkt
    prosáván zhruba 5 minut vzduchem. Produkt byl přenesen do odvážené $100 \: ml$ kádinky za pomoci
    kopistky. Kádinka společně s produktem byla pak odvážená ($m_{acetanilid}=0,8772\: g$,) a byl vypočten výtěžek (\emph{viz. Výpočty}).

    \subsection*{Rekrystalizace acetanilidu z horkého toulenu}
    Surový acetanilid o hmotnosti $1.00\: g$ byl přenesen do $25\: ml$ slzovité baňky. Baňka
    byla upevněna do klemy. Následně byla sestavena aparatura pro zahřívání pod zpětným chladičem.
    Skrze chladič bylo pomocí injekční stříkačky přidáno $10 \: g$ toulenu a směs byla zahřívána
    v olejové lázni k varu. Směs byla zahorka přefiltrována přes malý smotek vaty v plastové nálevce
    do Erlenmeyeřovy baňky o objemu $25 \: ml$. Směs byla následně zavřena plastovou zátkou, zchlazena
    pod proudem studené vody a vložena na 15 minut do mrazáku. Poté byl produkt odsán na fritě
    a promyt $2 \: ml$ toulenu. Zbylý produkt byl vysušován zhruba po dobu 5 minut. Do předem
    zvážené $100\: ml$ kádinky byl vyškrábán kopistkou produkt. V kádiňce byl následně produkt
    dalších 15 minut prosušován a následně odvážen a výtěžek vypočten ($m_{acetanilid}=0.6275 \: g$, \emph{viz. Výpočty}).

    \subsection*{Rekrystalizace síranu měďmatého změnou rozpouštědla}
    Byly naváženy $3.00g$ surového pentahydrátu síranu měďnatého, který byl následně rozetřen
    v třecí misce, aby jeho rozpouštění probíhalo lépe a předešlo se přidání přílišného množství
    rozpouštědla. Rozetřená modrá skalice byla přidána do $100 \: ml$ kádinky a poté byla
    postupně přidávána voda za míchání dokud se roztok nestal nasyceným. Do roztoku byla přidána
    jedna kapka $10\%$ kyseliny sírové a 1 lžička aktivního uhlí. Směs byla přefiltrována do $400\: ml$
    kádinky přes složený filtrační papír. Z přefiltrované směsi byl vysrážen produkt postupným
    přidáváním ethanolu ze střičky. Ethanol přestal být přidáván, až se matečný louh stal zcela bezbarvým.
    Sraženina byla odsána na fritě a promyta ethanolem. Následně byla vysušena ethanolem. Produkt, který
    představovala sraženina, byl odvážen ($m_{CuSO_4}=2,7260\: g$)  a výtěžek vypočten (\emph{viz. Výpočty})

    \section*{Výpočty}
    \subsection*{Rekrystalizace acetanilidu z horké vody}
    \begin{multicols}{2}
        \begin{center}
        \noindent$m_{produktu}=0.8772\: g$\\
        $m_1=3.00\:g$\\
        $\alpha=\frac{m_{produktu}}{m_1}$\\
        $\alpha=29\%$
        \end{center}
    \end{multicols}
    \subsection*{Rekrystalizace acetanilidu z horkého toulenu}
    \begin{center}
    \begin{multicols}{2}
        \noindent$m_{produktu}=0.6275\: g$\\
        $m_1=1.00\:g$\\
        $\alpha=\frac{m_{produktu}}{m_1}$\\
        $\alpha=63\%$
    \end{multicols}
    \end{center}
    \subsection*{Rekrystalizace síranu měďnatého změnou rozpouštědla}
    \begin{center}
    \begin{multicols}{2}
        \noindent$m_{produktu}=2.7260\: g$\\
        $m_1=3.00\:g$\\
        $\alpha=\frac{m_{produktu}}{m_1}$\\
        $\alpha=90\%$
    \end{multicols}
    \end{center}

    \section*{Závěr}
    Výtěžky nám, dle naměřených hodnot vyšly: 1. Rekrystalizace acetanilidu z horké vody
    $\to \alpha=29\%$; 2. Rekrystalizace acetanilidu z horkého toluenu $\to \alpha=63\%$;
    3. Rekrystalizace síranu meďnatého změnou rozpouštědla $\to \alpha=90\%$. Výtěžek při rekrystalizaci acetnalidu z horkého toulenu je větší, než
    výtěžek při rekrystalizaci acetnailidu z horké vody, protože v toulenu
    je rozpustných více nečistot ze vzorku, než ve vodě. Rekrystalizace změnou rozpouštědla je nejvýnosnější, protože v získané směsi dochází k překročení součinu rozpustnosti
    rouzpouštěné látky a tak se pak začne vylučovat v krystalech.

\end{document}
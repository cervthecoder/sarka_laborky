\documentclass[13pt, a4paper, twoside]{article}
\usepackage[utf8]{inputenc}
\usepackage{geometry}
\usepackage[czech]{babel}
\usepackage{chemformula}
\usepackage{chemfig}
\usepackage{enumitem}
\usepackage{fancyhdr}
\usepackage{setspace}
\usepackage{multicol}
\geometry{legalpaper, margin=1.05in}
\pagestyle{fancy}
\lhead{\Large Šárka Doležalová, skupina 6}
\rhead{\large 10.12.2020}
\begin{document}
\begin{center}
    \huge
    Úloha 6: Elektrolytická preparace a elektrogravimetrie
    \vspace{7mm}
\end{center}
\onehalfspacing
\large \noindent
\textbf{Zadané úlohy:}
\begin{enumerate}
    \item Anodickou oxidací roztoku síranu draselného v kyselině sírové připravte peroxodisíran draselný a proveďte s ním reakce podle návodu.
    \item Galvanickým pokovením naneste na kovovou katodu povrchovou vrstvu mědi; přitom ověřte platnost Faradayova zákona stanovením relativní atomové hmotnosti mědi.
    \item Poměděnou katodu použijte pro elektrogravimetrické stanovení molární koncentrace niklu v neznámém roztoku.
\end{enumerate}

\textbf{Teoretický úvod:}
\begin{itemize}
    \item Elektrolýza je redoxní děj, při kterém se na katodě ionty redukují a na anodě se oxidují. Peroxosíran si připravíme elektrolýzou, konkrétně oxidací síranu na anodě.
    \begin{center}
        $2SO_4^{2-} \to S_2O_8^{2-}+2e^-$
    \end{center}
    \item Galvanickým pokovováním se nám z vodných roztoků solí na katodě vylučuje kov, který rovnoměrně pokrývá celý její povrch.(Měď v našem případě) Pomocí Faradayova zákonu můžeme spočítat její relativní atomovou hmotnost.
    \item \emph{2. Faradayův zákon:} Množství různých látek přeměněná týmž nábojem jsou v poměru chemických ekvivalentů těchto látek.
    \begin{center}
        \Large
        $m=\frac{MIt}{zF}$
    \end{center}
    \item Elektrogravimetrii zjistíme hmotnost vyloučeného niklu. Z této hmotnosti spočítáme koncentraci nikelnatých iontů v původním roztoku.
    \begin{center}
        \Large
        $c=\frac{m}{MV}$
    \end{center}
\end{itemize}
\textbf{Postup:}\newline
V první části byla sestavena aparatura elektrolyzéru, dále připojena na bublačku odvádějící vznikající plyny a  směsí lihu a suchého ledu v Dewarově nádobě. Před začátkem elektrolýzy bylo chladící medium schlazeno na $-30^{\circ} C$, na této teplotě bylo i po celou dobu reakce udržováno. Zdroj napětí připojen k aparatuře byl nastaven na proud $1.5 A$ a napětí $6.0V$. Elektrolýza probíhala $65min$. Poté, co reakce skončila, byl elektrolyt odsán na fritě a produkt ($K_2S_2O_8$) byl několikrát promytý ethanolem od zbytku elektrolytu. Na analytické váze byla zvážena jeho hmotnost, $m=1.3541g$. Dosazením do rovnice 2. Faradayova zákonu jsme spočítali reálný výtěžek $K_2S_2O_8$ (viz. výpočty a chemické rovnice).

Výsledný produkt byl vložen do dvou zkumavek ($0.1g$), do jedné byly dále přidány
$3ml$ $5\%$ roztoku $KI$ a $0.5ml$ $10\%$ $H_2SO_4$ a do druhé $3ml$ roztoku $MnSO_4$ v $10\%$
a $0.5ml$ $5\%$ roztoku dusičnanu stříbrného. Obě zkumavky byly zahřány nad kahanem. Tyto pokusy byly poté zopakovány, ale
$K_2S_2O_8$ bylo nahrazeno $H_2O_2$ ($0.5ml$) (popis chemických reakce viz výpočty a chemické rovnice).

V druhé části úlohy byl kovový plíšek sloužící jako katoda aparatury pro elektrogravimetrii a galvanické pokovení odmaštěn pomocí
$6M$ $HNO_3$ a následně byl opláchnut destilovanou vodou a ethanolem, poté byl vysušen a zvážen na analytických vahách,
$m=15.3223g$. Následně byla sestavena aparatura pro galvanické pokovování s měděnými anodami a poměďovacím roztokem.
Proud byl nastaven na $0.5 A$ a napětí na $0.9V$. Reakci jsme nechali běžet $20min$.
Poté byla reakce zastavena, měděný plíšek, byl opět očištěn destilovanou vodou a ethanolem a usušen. Poměděný plíšek byl znovu zvážen na analytické váze,
$m=15.5608$. Podle 2. Faradayova zákonu jsme spočítali relativní atomovou hmotnost mědi. (viz výpočty a chemické rovnice)


V poslední části úlohy byla sestavena elektrogravimetrická aparatura s platinovými anodami a poměděným plíškem. Byly smíchány $10ml$
neznámého roztoku s číslem 61 se $120ml$ destilované vody, $5g$ pevného
$(NH_4)SO_4$. Elektrické napětí bylo nastaveno na $10V$. Reakce byla prováděno po dobu 
$75min$. Na hodinovém sklíčku jsme byli za použití diacetyldioximu přesvědčení, že byl z roztoku vyloučen všechen nikl. Kapka po smíchání s diacetyldioximem zrůžověla, ale netvořily se v ní žádné červené sraženiny. Aparatura byla odpojena od zdroje a byla vyjmuta poměděná destička s vyloučeným niklem. Tato destička byla očištěna destilovanou vodou a ethanolem a usušena. Po jejím zchladnutí byla zvážena analytických vahách,
$m=15.5943g$

\textbf{Vypočty a chemické reakce:}
\begin{itemize}
    \item \emph{Reálný výtěžek} $K_2S_2O_8$
    \begin{itemize}
        \item Elektrolýzou bylo připraveno $1.3541g$, teoretický výtěžek spočítán jako (2. Faradayův zákon):
        \begin{center}
            \Large
            $m=\frac{MIt}{zF}=\frac{270.32 \cdot 1.5 \cdot 3900}{2.96487}=6.2875g$
        \end{center}
        \item Reálný výtěžek je tedy:
        \begin{center}
            \Large
            $x=\frac{1.3541}{6.2875}=21.5\%$
        \end{center}
    \end{itemize}
    \item \emph{Chemické reakce s } $K_2S_2O_8$ \emph{a} $H_2O_2$
    \begin{enumerate}[label=\Alph*.]
        \item $2KI + K_2S_2O_8 + H_2SO_4 \to I_2 + 2K_2SO_4 + H_2SO_4$
        \\ výsledný roztok byl tmavě rudý, stoupaly fialové páry jodu
        \item $MnSO_4 + K_2S_2O_8 + 2H_2O \to K_2SO_4 + MnO_2 + 2H_2SO_4$
        \\ výsledný roztok byl černý, černá sraženina
        \item $H_2O_2 + 2KI + H_2SO_4 \to I_2 + K_2SO_4 + 2H_2O$
        \\ výsledný roztok byl tmavě žlutý, černá sraženina, stoupaly fialové páry jodu
        \item $H_2O_2 + 2KI + H_2SO_4 \to$ reakce neprobíhá
        \\ Výsledný roztok byl bezbarvý.
    \end{enumerate}
    \item \emph{Relativní atomová hmotnost $Cu$}
    \\ Galvanickým pokovením vzniklo na plíšku $0.2385g$ mědi, spočítáme relativní atomovou hmotnost mědi (2. Faradayův zákon):
    \begin{center}
        \Large
        $A_r = \frac{mzF}{It} = \frac{0.2385\cdot 2 \cdot 96487}{0.5 \cdot 1200}=76.707$
    \end{center}
    \item \emph{Koncentrace nikelntatých iontů}
    \\ Galvanickým pokovením vzniklo na plíšku $0.0335g$ niklu, spočítáme koncentraci nikelnatých iontů:
    \begin{center}
        \Large
        $c=\frac{m}{MV}=\frac{0.0335}{58.69\cdot 0.01}=0.057mol\cdot dm^{-3}$
    \end{center}
\end{itemize}
\textbf{Závěr:}\\
Praktický výtěžek $K_2S_2O_8$ bylo proti teoretickému jen $21.5\%$.
Z pozorování reakcí s $K_2S_2O_8$ a $H_2O$, jsme vyvodili, že $K_2S_2O_8$ 
je silnějším oxidačním činidlem. Vypočtená relativní atomovou hmotnost mědi
$A_r=76.707$ se nám významně mění od té tabulkové $A_r=63.55$. Tato pravděpodobně vznikla nepřesným měřením času reakce.
Koncentrace nikelnatých iontů v neznámém roztoku číslo 61 nám vyšla $c=0.0570 mol\cdot dm^{-3}$.



\end{document}